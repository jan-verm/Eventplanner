\documentclass[a4paper,11pt]{article}

\usepackage[dutch]{babel}
\usepackage{url}
\usepackage[latin1]{inputenc}
\usepackage{fullpage}
\usepackage{xspace}

\begin{document}

\pagenumbering{roman}

\title{Reflectie\\Practicum 2\\Internettechnologie}
\author{Maxime Fern\'andez Alonso\\Jan Vermeulen}
\date{}
\maketitle

\pagenumbering{arabic}

\subsection*{HTML-toepassingen vs. native toepassingen}
Het grootste verschil tussen beide is dat een native toepassing als standalone toepassing gedownload moet worden en dat de HTML-toepassing ten allen tijde direct toegankelijk is. Ik denk dat het grootste nadeel aan een native toepassing de tijd is die het kost om te developen. Een HTML-toepassing kan veel minder tijd en moeite kosten indien er al een professionele webapplicatie voor bestaat. Ook updates en/of fixes laten uitrollen zijn gemakkelijker bij een HTML-toepassing, op een native toepassing moeten updates manueel ge\"installeerd worden. Dit zorgt ervoor dat sommige gebruikers niet altijd de nieuwste versie van de applicatie aan het gebruiken zijn.\\
Een native toepassing heeft natuurlijk ook zijn voordelen. Deze zal hoogstwaarschijnlijk beter presteren dan een HTML-toepassing. Aangezien men gebruik maakt van een professionele SDK zal het navigeren binnen de applicatie typisch veel eenvoudiger en aangenamer zijn voor de gebruiker. Een verschil dat we nog niet zouden merken in dit practicum is ook dat een applicatie gebruik kan maken van functies, zoals de camera, van een toestel, terwijl dit niet mogelijk is met een HTML-toepassing.\\
Over het algemeen is het gemakkelijker een HTML-toepassing te maken, terwijl het beter is, voor zowel de gebruiker als de provider, om een native toepassing aan te bieden. 

\end{document}

